% !Mode:: "TeX:UTF-8"

\markboth{总结与展望}{总结与展望}
%\addcontentsline{toc}{chapter}{结\quad 论} %添加到目录中
\chapter{总结与展望}

本文关注于动态复杂网络社团检测问题,依托于随机块模型框架,从模型的角度研究动态网络中的社团演化问题,同时探究了影响动态网络中节点社团转移的结构属性。随后从城市风险计算的角度对本文的方法以及探究的规律的有效性进行了实证分析。本章对本文的研究内容和研究成果进行总结分析,并指出其中模型的不足之处与未来的研究思路。
\section{总结}
社团结构作为复杂网络的重要任务之一,对于理解复杂网络结构形成、功能探索和领域应用如风险计算有重要的作用。而复杂网络的动态演化给社团检测带来了新的挑战,包括社团结构的演化、社团的产生与消亡、社团的分裂与合并等问题,与此同时,动态网络的社团检测也迫切需要我们对社团的演化行为进行建模。作为复杂网络社团检测的重要方法之一,随机块模型对网络的生成机制解释性的非常好,可以有效地建模复杂网络中社团的形成机制。而基于随机块模型构建的动态网络模型能够在有效地建模网络形成机制的同时把握网络中的社团演化机制。本文立足于动态随机块模型框架,从模型的角度对动态网络的社团结构进行建模,旨在解决动态网络中节点社团演化机制的问题。于此同时,本文还对节点的结构属性与节点的社团归属演变之间的关系通过多个真实数据集进行了探究,得到了对社团演化机制探究至关重要的结论。在应用方面,本文利用手机信令数据验证了本文提出的模型对城市风险计算的有效性。本文的主要工作和贡献点总结如下:
\begin{itemize}
	\item 本文利用15个真实复杂网络数据集,包括社交网络数据(如twetter、facebook等)、wiki数据及社区通话数据等,通过TILES方法对数据进行动态网络社团检测,同时利用决策树将相邻时间片的节点是否发生社团转移作为二分类标签,将节点的结构属性通过特征工程组合为决策树的分类特征对节点进行二分类。通过对决策树分类后的分类模型中的特征重要性进行计算,本文得出结论:节点的度和节点的平均邻居度对节点的社团归属变化影响最大。并且在后续的案例分析中,本文找到了与之相佐证的真实情景。
	\item 基于动态随机块模型(DSBM),本文提出了层次贝叶斯动态随机块模型(HB--DSBM)。该模型在相邻时间片引入了节点级别的社团转移参数,并提出了层次贝叶斯结构来生成社团级别的节点社团转移参数和节点级别的社团转移参数,并利用更细粒度的节点级别社团转移参数生成节点的在动态网络的社团归属。同时本文还对HB-DSBM提出了高效的变分推断算法,通过变分推断算法来对模型近似求解,同时提升了其算法运行效率,比之传统的DSBM模型运行效率更能适应大规模数据。通过将HB-DSBM和不同类的动态社团检测方法进行对比,并进行了社团演化分析对比,结果显示HB-DSBM在动态社团检测效果高于同类方法的同时,受益于更细粒度的社团转移参数,对于动态网络的社团演化分析更加精准。
	\item 通过对手机信令数据的处理,结合天津地块信息构建了多层复杂网络,并利用HB-DSBM对手机信令数据进行社团划分,并进一步通过社团标签-事件的提取方法计算出事件以及事件发生地以及事件发生时间。随后根据相关文献,通过合理的设计打分策略评判出每个事件的风险程度,并结合实际评判其合理性,以证明文章提出的HB-DSBM在风险计算中的可行性以及有效性。
\end{itemize}

\section{展望}
本文对于节点结构特征对社团演化的影响的探究以及构建的HB-DSBM模型对动态复杂网络社团演化的探究具有一定的贡献,但是随着最新的动态网络社团检测建模发展趋势以及城市风险计算发展的影响,本文的许多工作需要我们进行改进或者进一步研究。同时针对城市风险计算的需求,本文利用HB-DSBM对天津市手机信令数据进行社团检测,并利用相关文献进行事件提取以及每个事件的风险值评估,得到了风险事件的结果,然而该实证的计算依然存在一些待完善的部分。具体如下:
\begin{itemize}
	
	\item 对于节点结构特征对社团演化的影响的探究中,我们得出结论:节点的度和节点的平均邻居度对节点的社团关系变化影响至关重要。其中,节点的度对节点的社团关系变化的影响是显而易见的,同时目前也有很多方法将节点的度应用于社团检测中来增强社团检测的效果,这些方法也都取得了预期的结果,这侧面证明了节点的度在动态网络社团检测中的重要作用。然而,节点的平均邻居度对节点社团转移的影响的内在规律还需要我们进一步探究,相信在不就的将来,节点的平均邻居度也可以有效的作用于动态网络社团检测。
	\item HB-DSBM模型在融合了节点粒度的转移参数后,确实对动态网络社团检测以及社团演化分析都起到了很大的作用,但是该模型在引入了如此细粒度的参数后,使得其参数空间变得非常大,虽然本文提出了变分推断以及在实现中使用了随机采样等策略一定程度上降低了算法的复杂度,但是仍然达不到真正应用于生产的程度。因此对模型适当改进以降低其参数规模是HB-DSBM的下一步改进方向。
	\item HB-DSBM模型改进自DSBM,这使得其继承了DSBM的一大缺陷,即该模型不能适用于重叠社团的检测。在现实世界大部分情景中,同一个节点可以属于多个社团,例如大学学生可以加入多个兴趣社团等。该问题在静态网络随机块模型中得到了较好的解决,即混合随机块模型。而针对动态随机块模型的重叠社团问题,也已经有一些方法,但是均存在一些缺陷,因此对HB-DSBM进行适当改进使得其能够检测重叠社团也是其下一步重要的改进方向之一。
	\item HB-DSBM模型的一大假设就是需要预先知道动态网络中社团的个数,这在真实数据中是不可行的,例如在社交网络中,兴趣小组的个数不可能在不经聚类之前就提前知晓。同类方法的处理方式一般为在计算时设置社团个数$K=\log N$,其中,$N$为网络中的节点个数。再通过适当的方法将没有实际意义的社团舍弃以达到模型选择的目的(例如仅有一个节点的社团),而此种方法会大大增加算法的计算量,因此也不是很好的模型选择方法。HB-DSBM继承自DSBM,因此该方法的模型选择可以参考现有的针对DSBM的模型选择方法进行改进,如利用狄利克雷过程利用数据自动确定社团个数等。因此通过对HB-DSBM进行适当的改进来使该模型能够根据数据特征自动确定社团个数,即模型选择,也是其下一步改进的方向之一。
	\item 本文在实证部分,仅融合了手机信令数据与天津市地块数据构建了二层复杂网络对事件进行提取并对每个事件进行了风险分析。而真正的城市风险计算需要融合地块、事件、人员、人员行为等包括物理空间和网络空间的数据进行多层次融合以及计算之后才能得出准确的风险计算结果,多源信息融合也是风险计算中进行风险预测、风险决策、风险管控的基础,因此对于风险计算部分,下一步需要改进的就是完善数据收集以及信息融合的底层架构,这样才能在未来进行更深层次的风险计算。
\end{itemize}
